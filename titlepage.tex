\font\title=\fontname\tenbf\space scaled 4000
\hbox{}
\vskip 0pt plus 1fil\relax
\hbox to \hsize{\hss\title Vikiho zpěvník\hss}
\vskip 0pt plus 2.5fil\relax
\hbox to \hsize{\hss Vygenerováno \the\day. \the\month. \the\year}
\eject
\hbox{}
\vskip 0pt plus 1fil\relax
{
\parindent=20pt
\parskip=5pt
Když jsem bydlel na koleji 17. listopadu, pravidelně jsme se scházeli
ke zpěvu na schodech, obvykle kolem 13. patra budovy B. Zpívali jsme většinou ze zpěvníku
Absolutní Abstinenti. Ten ale obsahuje jen přibližně tři sta písní, a tak netrvalo dlouho,
a většina z nich byla ozpívaných. Zpívat jiné věci ale bohužel příliš nešlo -- vždy by mohla
zpívat pouze malá část přítomných, kteří měli počítače nebo si pamatovali text. Někteří
lidé tento problém řešili nošením vlastních vytištěných akordů a textů k písním, které si chtěli zazpívat.
Nejdále tento nápad dovedl Karel Tuček, který nejen vytvořil zpěvník se vším všudy, ale dokonce k němu dodal
i~skripty, s~pomocí kterých si mohl vlastní zpěvník poskládat kdokoli.

Těch jsem v~roce 2018 využil a vytvořil svůj první zpěvník. Bohužel byl sám o~sobě příliš tenký, v~kombinaci
s~Karlovým potom příliš tlustý a současně obsahoval spoustu písní, které jsem hrát neuměl nebo mě nebavily. Přesto jsem zkombinovaný zpěvník nějakou dobu používal.
Když se mi potom v roce 2022 začal i~fyzicky rozpadat, zařekl jsem se, že si vyrobím nový. Chtěl jsem však, aby nový
zpěvník používal má vlastní makra, a~proto vznikl až roku následujícího.

Protože se však měl se svou první verzí netriviálně překrývat, rozhodl jsem se, že vstupní formát bude víceméně
kompatibilní s původním zpěvníkem. To vedlo k~tomu, že jsem nemusel přepisovat všechny písně znovu. To má bohužel
jeden vedlejší efekt -- původní zpěvník jsem začal psát přibližně ve stejné době, kdy jsem se začal učit hrát na
kytaru. Neuměl jsem tenkrát ještě dost rychle číst akordy s~béčky, tak jsem je přepsal na odpovídající akordy s~křížky.
Stopy tohoto řádění lze tedy bohužel najít i~ve zpěvníku, který držíš v~ruce (a není v~plánu takto postižené písně opravovat).

Ve zpěvníku lze písně převzaté z~původního zpěvníku od nových odlišit podle čísla -- staré mají černé na bílém pozadí, nové opačně.
}
\eject
